\documentclass[12pt]{extarticle}
\usepackage[english]{babel}
\usepackage[utf8]{inputenc}
\usepackage{amsmath}
\usepackage{graphicx}
\usepackage[colorinlistoftodos]{todonotes}
\usepackage[hidelinks]{hyperref}
\usepackage{courier}
\usepackage{booktabs}
\usepackage[T1]{fontenc}
\usepackage{tabularx}
\usepackage[margin=0.5in]{geometry}
\usepackage{breakurl}
\fontencoding{T1}
\fontfamily{garamond}
\fontsize{12}{15}
\selectfont

\begin{document}
\begin{center}
\bf \Huge BoxIn\\
Developer Guide
\includegraphics[width=\textwidth]{BoxIn}
\end{center}



\tableofcontents

\newpage
\section{Background}

BoxIn is a C++ based to-do list manager and this guide will explain how it works. The design of BoxIn adheres to the following coding standards:

\subparagraph{SLAP}    
Under SLAP (Single Level Abstraction Principle), code should be well abstracted so that each function only has one level of function calls. This prevents code from becoming convoluted by abstracting away the details of how a function is implemented

\subparagraph{Memory Management}
Coding in C++ requires efficient use of memory. Memory should be freed up if it is no longer in use

\subparagraph{Coding standard}
BoxIn code follows the coding standard listed out here:\\ \texttt{http://tinyurl.com/BoxInCodingStandard}

\subparagraph{Namespaces}
All namespaces should be marked out to avoid confusion with the Boost library which is used in many parts of the code. cf. \texttt{std::string} rather than simply \texttt{string}\\

\section{Anatomy}

\includegraphics[width=\textwidth/2]{architecture}\\

BoxIn has four major components, the GUI, the Logic, the Storage and the Parser components. The division of these components is done in a way that follows two guiding principles. We also apply the Model View Controller pattern.

\subsection{Model View Controller pattern}
In BoxIn, the GUI component acts as both the View and the Controller. Users view all events through the GUI and the GUI is also responsible for taking care of all user interaction, including mouse clicks and information sent through the command line. More details are found in Section 3.
The Model in the system is the Event class. More details are found in Section 5.

\subsection{Separation of Concerns}
Each major component of the application handles it's own concerns. For example, the GUI or Storage components do not process any command line input, only the Logic and Parser components do.

\subsection{Law of Demeter}
Classes which do not have any direct relation with each other should not be calling each other. This law is particularly effected in the fact that the GUI and the Storage classes have no knowledge of each other - neither calls any functions of the other.


\section{GUI}
\includegraphics[width=\textwidth / 2]{GUI_class_diagram}
\includegraphics[width=\textwidth / 2]{gui_sequence_diagram}



The GUI component acts as both the controller and the view in the MVC pattern. The library used to display the graphics is the Qt library. The documentation for the Qt library is available here:\\ \href{http://qt-project.org/doc/}{http://qt-project.org/doc/}\\
The above diagrams give a graphical representation of how the GUI is designed in both the sequential calls on user actions and as a component.
The GUI is divided into 7 components, discussed separately below:

\subsection{BoxIn (main window)}
The \texttt{BoxIn} class is the main window. All sub-components found in this window should also have this window as a parent window. This class inherits from \texttt{QWidget}. The \texttt{BoxIn} class mainly acts as a container for most of the GUI components.\\
The above sequence diagram shows the generic flow of events within the GUI component every time the user presses the Return key.\\
\subparagraph{Private attributes}
\begin{tabular}{l r}
Attribute type & Name\\
\hline
Ui::BoxInClass & ui\\
Logic & logic\\
QAction* & minimizeAction\\
QAction* & restoreAction\\
QAction* & quitAction\\
DigitalClock & clock\\
QLabel* & nameLabel\\
QLabel* & placeLabel\\
QLabel* & startLabel\\
QLabel* & idxLabel\\
QSystemTrayIcon* & trayIcon\\
QMenu* & trayIconMenu\\
DisplayFeed* & displayFeedIdx\\
QLineEdit* & commandLine\\
\end{tabular}
\subparagraph{Public methods}
\begin{tabular}{l r}
Return type & Method\\
\hline
void & displayFeedback(QString feedback)\\
void & clearCommandLine()\\
QString & readCommandLine()\\
void & setVisible(bool visible)\\
void & updateGUI()\\
void & createComponents()\\
void & setComponentSizes()\\
void & setComponentColors()\\
void & linkEvents()\\
void & createTrayIcon()\\
void & createActions()\\
void & iconActivatd()\\
void & changeEvent(QEvent *event);\\
\end{tabular}\\

\subparagraph{Key API}
The following are the key methods that deal with the functionality of the GUI\\
\begin{tabular}{p{6cm} p{12cm}}
Method & Description\\
\hline
returnPressed() & This signal is by the Qt framework whenever the user presses return with the command line in focus\\
linkEvents() & This method is a setup method linking all the relevant signals to the respective slots for processing\\
commandLineReturnPressed() & The slot connected to the returnPressed() signal mentioned above. This starts the chain of events which result in event processing, by reading the line and sending it to the Logic component\\
readCommandLine() & Returns the string held by the command line. Called exclusively by commandLineReturnPressed()\\
clearCommandLine() & Removes any text in the commandLine. Also called exclusively by commandLineReturnPressed()\\
displayFeedback(QString feedback) & Calls setText() on the FeedbackLine to display feedback\\
\end{tabular}

\subparagraph{Application icon}
The entire application has a predefined icon initialized in the constructor of BoxIn and packaged together with the application. The entire program can be minimized to the System Tray.\\
\begin{tabular}{p{6cm} p{12cm}}
Method & Description\\
\hline
createActions() & Creates the actions achievable by right-clicking the system tray icon. The actions supported are Minimize, Restore and Quit. These actions are then connected to the relevant slots to apply them\\
createTrayIcon() & Creates the icon itself and the supported menu, adding items in\\
setVisible() & Handles the minimize / maximize actions\\
iconActivated() & Handles the double-click event from the user.
\end{tabular}

\subparagraph{Fixed Size}
The \texttt{BoxIn} main window is of a fixed size (1000 x 600). This size is implemented through the constants \texttt{WIDTH\_WINDOW} and \texttt{HEIGHT\_WINDOW}.

\subsection{DisplayFeed}
The \texttt{DisplayFeed} inherits from \texttt{QListWidget}. This widget contains data members of type QEventStore, which make up the View component of the MVC design pattern. \texttt{DisplayFeed} is designed in its' own constructor, without a \texttt{.ui} file. \texttt{DisplayFeed}'s purpose is to display all events the user wishes to view. At the moment, it simply displays everything.
\subparagraph{Public methods}
\begin{tabular}{c c}
Return type & Method\\
\hline
void & addItem(QListWidgetItem* item) (Inherited)\\
void & setBorder()\\
void & refresh(std::vector<Event*> *thingsToInclude)\\
void & setItemColors()\\
std::string & pad(std::string str, int spaces)\\
std::string & reprDate(std::string date)\\
std::string & formatEvent(Event* event)\\
\end{tabular}

\subparagraph{Key API}The DisplayFeed uses the following key methods to display input\\
\begin{tabular}{p{6cm} p{12cm}}
Method & Description\\
\hline
refresh() & This is the function call made by BoxIn's commandLineReturnPressed(). This sets off the chain of other methods used to display the input. It creates a QEventStore pointer for each item to display and adds them to its internal display\\
formatEvent() & This method takes an event, extracts its data and turns it into a equally-spaced string representation of the event\\
setItemColors() & Changes the text color for the items - red for past and not-yet-done items, purple for the latest change and the rest alternate between black and grey so as to differentiate rows\\
pad() & Adds whitespace or truncates overly long strings to give even sizing\\
reprDate() & Replaces dates with Today / Tomorrow for the matching dates
\end{tabular}

\subsection{FeedbackLine}
The FeedbackLine inherits from QLabel and is a simple instant feedback system for the user. It simply displays messages coming from the Logic component regarding the success or failure of user commands.

\subparagraph{Key API} This object only implements one important method\\
\begin{tabular}{p{6cm} p{12cm}}
Method & Description\\
\hline
setText(QString feedback) & Sets the text on the feedback line to the given input\\
\end{tabular}

\subsection{CommandLine}
The \texttt{CommandLine} component of the GUI is the controller for majority of the system. Since the target audience prefers to use a command line style input, this becomes the main input interface. This component inherits from \texttt{QLineEdit}. Its key API is discussed at a wider level with BoxIn above.\\

\subsection{DigitalClock}
\texttt{DigitalClock} is simply a digital clock displayed on the main window. It tells the time with a flashing colon. This object inherits from \texttt{QLCDNumber}\\

\subsection{QEventStore}
\texttt{QEventStore} is the wrapper class for the \texttt{Event} class implemented. This class allows \texttt{Event} objects to be added to the \texttt{DisplayFeed} so that a direct association is kept between the objects in the \texttt{DisplayFeed} and the \texttt{Event} objects themselves. \texttt{QEventStore} inherits from \texttt{QListWidgetItem}

\subparagraph{Key API} This class only implements one key method.\\
\begin{tabular}{p{6cm} p{12cm}}
Method & Description\\
\hline
getEvent() & This function takes any information available from the stored event and returns a \texttt{QString} representation of it.
\end{tabular}

\subsection{QHelpWindow}
This window provides an interface for the user to view examples and various help options regarding the usage of BoxIn. It is created by the \texttt{Logic} component when the user enters the command \texttt{help}.\\
The \texttt{QHelpWindow} contains a QComboBox which the user uses to select a function he wishes to view help for, and the \texttt{currentIndexChanged()} signal is emitted and caught by the \texttt{QHelpWindow} to change the text contained in the \texttt{QTextEdit}.\\


\section{Logic}
\includegraphics[width=\textwidth /2, height=\textheight / 3]{logic_class_diagram}
\includegraphics[width=\textwidth /2, height = \textheight / 3]{add_call_diagram}


The logic of the system is explained by the above sequence diagram as an example call to the add functionality of the system. The user input triggers the add command and the add command is passed to the storage to be executed.\\
The Logic component is made up of 2 major classes - the Logic class and the Action class (which is subclassed out to different actions).

\subsection{Logic}
The \texttt{Logic} class is the main director of movement. Based on input given through the GUI, it redirects the system to take the appropriate actions so as to produce the correct result. It is also responsible for filtering irrelevant events from the GUI.
\subparagraph{Type Definitions}
\begin{tabular}{c p{9.5cm}}
Type Name & Purpose\\
\hline
CommandType & Handles input according to the CommandType, which is also extracted in within \texttt{Logic}\\
FilterType & Handles the filtering based on the type of filter requested by the user. The default filter is to have no filter (\texttt{FilterType::None})
\end{tabular}
\subparagraph{Private attributes}
\begin{tabular}{c c}
Attribute Type & Name\\
\hline
SimpleStorage & storage\\
std::map<std::string, CommandType> & stringToCommand\\
std::map<std::string, Filter::FilterType> & stringToFilter\\
SimpleParser & parser\\
Filter::FilterType & filter\\
\end{tabular}
\subparagraph{Public methods}
\begin{tabular}{c c}
Return Type & Method\\
\hline
void & setupMap()\\
std::string & handleUserInput(std::string input)\\
std::vector<Event*> & getEvents()\\
\end{tabular}
\subparagraph{Key API}
The Logic class implements the following methods\\
\begin{tabular}{p{6cm} p{12cm}}
Method & Description\\
\hline
handleUserInput() & Processes a user's input, creates the appropriate action and passes the action to Storage for execution\\
getEvents() & Returns the events filtered by filter as a vector\\
\end{tabular}

\subsection{Action}
The Action class is an interface for the various possible executable actions for BoxIn. These actions include \texttt{Add}, \texttt{Delete}, \texttt{Edit} and \texttt{Mark}.\\

\subparagraph{Private attributes}
\begin{tabular}{c c}
Attribute Type & Name\\
\hline
SimpleParser & parser\\
\end{tabular}
\subparagraph{Public methods}
\begin{tabular}{c p{11cm}}
Return Type & Method\\
\hline
std::string & execute(std::vector<Event*>\&)\\
std::string & undo(std::vector<Event*>\&)\\
void & deleteEvent(std::vector<Event*> \&events, Event* event)\\
Event* & findEventByIdx(int idx, std::vector<Event*> \&events)\\
Event* & findEventByNameAndEndDate(std::string name, std::string endDate, std::vector<Event*> \&events)\\
\end{tabular}
\subparagraph{Key API}
All Action objects follow the following API, though implementation may vary slightly. Refer also to the class diagram at the top of this section to understand the inheritance structure.\\
\begin{tabular}{p{6cm} p{12cm}}
Method & Description\\
\hline
execute() & Declared virtually in \texttt{Action} and implemented in its sub-classes. Applies the \texttt{Action} onto the vector of \texttt{Event} pointers\\
undo() & Declared virtually in \texttt{Action} and implemented in its sub-classes. Does the exact opposite of \texttt{execute}\\
deleteEvent() & Deletes a given \texttt{Event} pointer in the \texttt{Event*} vector\\
findEventByIdx() & Returns the \texttt{Event} pointer associated with the given integer index\\
findEventByNameAndEndDate() & Returns the \texttt{Event} pointer associated with the given name and end date as strings\\
\end{tabular}




\section{Parsers}
The parsers for BoxIn deal with extracting information out of a user-given string. There are two parsers used in BoxIn - the SimpleParser (for generic items) and the TimeParser, which deals exclusively with times

\subsection{SimpleParser}
The SimpleParser deals more with dates and basic parsing. The following date formats are accepted: \texttt{DDMMYY, YYYYMMDD, YYYY/Jan/DD, monday, tuesday etc., today, tomorrow}.\\

\subparagraph{Type Definitions}
\begin{tabular}{c c}
Type Name & Purpose\\
\hline
InfoType & Determines the information to be extracted\\
DateFormat & Matches the date format to the correct parsing algorithm\\
\end{tabular}

\subparagraph{Private attributes}
\begin{tabular}{c c}
Attribute type & Name\\
\hline
std::map<InfoType, std::string> & keywordMap\\
std::map<std::string, std::string> & monthMap\\
std::map<std::string, boost::date\_time::weekdays> & dayMap\\
\end{tabular}

\subparagraph{Public methods}
\begin{tabular}{c c}
Return Type & Method\\
\hline
std::string & getField(std::string input, InfoType info)\\
void & setupMaps()\\
bool & isKeyword(std::string word)\\
bool & isInteger(std::string text)\\
boost::gregorian::date & convertToDate(std::string date)\\
DateFormat & matchFormat(std::string date)\\
bool & isNumericalFormat(std::string date)\\
bool & isDayOfWeek(std::string day)\\
bool & isToday(std::string day)\\
bool & isTomorrow(std::string day)\\
std::string & removeEscapeChar(std::string word)\\
std::string & removeWhitespace(std::string text)\\
\end{tabular}

\subparagraph{Key API}The \texttt{SimpleParser} implements the following key API to extract data. Many of the functions are used to match dates\\
\begin{tabular}{p{6cm} p{12cm}}
Method & Description\\
\hline
getField() & Retrieves the information matching the InfoType provided by the caller\\
isKeyword() & Returns \texttt{true} if the word given is a keyword\\
removeEscapeChar() & Returns the word removing the escape character \texttt{.}\\
removeWhitespace() & Trims trailing whitespace on a string\\
convertToDate() & Converts a string into a boost::gregorian::date object by matching formats using the other functions. If a match is not found, returns boost::gregorian::not\_a\_date\_time\\
matchFormat() & Returns the format which matches the string it was given. If an appropriate format is not found, returns \texttt{FormatNotRecognised}\\
\end{tabular}

\subsection{TimeParser}
The TimeParser deals particularly with the parsing of times from user strings. The following formats are accepted: \texttt{HHMM, HH:MM}

\subparagraph{Key API}
The TimeParser only implements one key method\\
\begin{tabular}{p{6cm} p{12cm}}
Method & Description\\
\hline
convertToTime() & Converts a string into a boost::posix\_time::ptime object by identifying it by length. Returns boost::date\_time::not\_a\_date\_time
\end{tabular}



\section{Storage}
\includegraphics[width=\textwidth/2]{storage_class_diagram}\\
The Storage class keeps both the internal representaion and the \texttt{json} representation of the Event classes. There are 3 major components, discussed below:

\subsection{SimpleStorage}
SimpleStorage is the highest level structure of the Storage component of BoxIn. It handles all the interactions with the Logic component. The \texttt{json} storage file is declared as a constant here as \texttt{BoxInData.json}
\subparagraph{Type Definitions}
\begin{tabular}{c c}
Type Name & Purpose\\
\hline
SortCriteria & Determines the sorting method\\
\end{tabular}
\subparagraph{Private attributes}
\begin{tabular}{c c}
Attribute type & Name\\
\hline
std::vector<Event*> & events\\
std::stack<Action*> & actionStack\\
FileStorage & file\\
SortCriteria & criteria \\
\end{tabular}
\subparagraph{Public methods}
\begin{tabular}{c c}
Return Type & Method\\
\hline
std::vector<Event*> & getEvents()\\
void & pushStack(Action* action)\\
Action* & popLastAction()\\
void & sortEvents()\\
std::string & execute(Action* action)\\
std::string & undo(Action* action)\\
void & saveFile()\\
\end{tabular}
\subparagraph{Key API}
SimpleStorage implements the following key API to store and undo events. These are called from the Logic component. In addition, the FileStorage is read in the initializer.\\
\begin{tabular}{p{6cm} p{12cm}}
Method & Description\\
\hline
getEvents() & Returns all events found in Storage\\
pushStack() & Adds an executed Action to the stack of things done\\
popLastAction() & Removes the most recent Action and returns it. Used in \texttt{undo}\\
sortEvents() & Arranges all the events in \texttt{events} by their end date\\
execute(Action* action) & Executes action onto \texttt{events} (see Action class for details)\\ 
undo(Action* action) & Undo-es action from \texttt{events}\\
saveFile() & Calls the FileStorage class to write all events to the \texttt{json} file\\
\end{tabular}

\subsection{Event}
The Event class stores data of one Event. It is the model class being passed around the entire system and is therefore included in almost every component.
\subparagraph{Private attributes}
\begin{tabular}{c c}
Attribute Type & Name\\
\hline
std::map<std::string, Field> & fieldMap (used in generic edit)\\
std::string & name\\
boost::gregorian::date & sdate (start date)\\
boost::gregorian::date & edate (end date)\\
boost::posix\_time::ptime & stime\\
boost::posix\_time::ptime & etime\\
std::string & nonformattime\\
std::string & location\\
SimpleParser & parser\\
TimeParser & timeParser\\
int & idx\\
bool & recent\\
bool & done\\
\end{tabular}
\subparagraph{Public methods}
\begin{tabular}{c p{8.5cm}}
Return Type & Method\\
\hline
Event* & copy()\\
std::map<std::string, Field> & setupMap()\\
std::string & getName()\\
std::string & getStartDate()\\
std::string & getEndDate()\\
std::string & getStartTime()\\
std::string & getEndTime()\\
std::string & getLocation()\\
int & getIdx()\\
bool & isRecent()\\
boost::posix\_time::ptime & getPosixStartTime()\\
boost::posix\_time::ptime & getPosixEndTime()\\
void & editField(std::string field, std::string newValue)\\
void & setName(std::string newName)\\
void & setStartDate(std::string newDate)\\
void & setEndDate(std::string newDate)\\
void & setStartTime(std::string newTime)\\
void & setEndTime(std::string newTime)\\
void & setLocation(std::string newLocation)\\
void & setIdx(int newIdx)\\
void & removeRecent()\\
void & setDone(bool newValue)\\
bool & getDone()\\
boost::gregorian::date & getDateFromInput(std::string date, std::string time)\\
boost::gregorian::date & getDateFromInput(std::string date,  std::string time, std::string preDate)\\
boost::posix\_time::ptime & getTimeFromInput(boost::gregorian::date date, std::string time)\\
std::string & repr()\\
\end{tabular}

\subparagraph{Key API}
The Event class implements the following key API which helps it store data accurately.\\
\begin{tabular}{p{6cm} p{12cm}}
Method & Description\\
\hline
editField() & Identifies the correct field to edit and from there calls the appropriate function. This allows for a more generic edit call\\
repr() & Gives a textual representation of the data in the event. Used for testing purposes, but not in the main program\\
getDateFromInput() & Makes use of the SimpleParser to convert a date to the appropriate date, failing which will put the default date (today if none is specified) if there is a time associated to the date. Used in the constructor, and overloaded for the case where there is no default date\\
getTimeFromInput() & Returns the appropriate boost::posix\_time::ptime object that tallies with the associated date and time given. Used in the constructor\\
\end{tabular}

\subsection{FileStorage}
The FileStorage class deals exclusively with reading and writing files to and from a \texttt{json} file.
\subparagraph{Private attributes}
\begin{tabular}{c c}
Attribute Type & Name\\
\hline
std::string & filename\\
\end{tabular}

\subparagraph{Public Methods}
\begin{tabular}{c c}
Return Type & Method\\
\hline
void & saveFile(std::vector<Event*>)\\
void & writeEvent(json\_spirit::Array \&eventArray, Event* event)\\
std::vector<Event*> & readFile()\\
Event* & readEvent(const json\_spirit::Object\& obj, unsigned int idx)\\
\end{tabular}

\subparagraph{Key API}All methods are critical to the correct reading and writing of the \texttt{json} file.\\
\begin{tabular}{p{6cm} p{12cm}}
Method & Description\\
\hline
saveFile() & Takes in a vector of events and writes all of them to the \texttt{json} file\\
writeEvent() & Takes in an json\_spirit Array and a single Event pointer and adds that pointer to the Array\\
readFile() & Returns a vector of Event pointers stored in the \texttt{json} file\\
readEvent() & Reads and creates a single Event pointer from the \texttt{json} file as a json\_spirit Object\\
\end{tabular}

\newpage

\section{Appendix A}

\subsection{Use cases}
\begin{tabular}{l|l}
Name                  & UC01:Add a new task                                                                                                                                                                                                                \\
Description           & To add a new task                                                                                                                                                                                                                  \\
Pre Condition          & BoxIn is currently running                                                                                                                                                                                                         \\
Basic course of event & \begin{tabular}[c]{@{}l@{}}1. User indicates the event that they want to add (Name, Date, Time, Place)\\and it has to be in this specific order\\ 2. BoxIn will give feedback indicating that the event has been added\end{tabular} \\
Alternative path      & \begin{tabular}[c]{@{}l@{}}1. One of the parameters is missing\\      1a. BoxIn responds that a parameter is missing and ask the user to try again\end{tabular}                                                                \\
Post Condition        & A new event is added and saved.                                                                                                                                                                                                   
\end{tabular}
\linebreak
\\[1cm]
\begin{tabular}{l|l}

Name                  & UC02: Delete a task                                                                                                                        \\
Description           & To delete an existing task.                                                                                                                                        \\
Pre Condition         & BoxIn is already running.                                                                                                                                          \\
Basic Course of Event & \begin{tabular}[c]{@{}l@{}}1. User types the command to delete an already existing task.\\ 2. The program deletes the task as per the user's command.\end{tabular} \\
Alternative Path      & \begin{tabular}[c]{@{}l@{}}1. If the task does not exist, the program displays the relevant message.\\ 2. Prompts the user to re-enter the command.\end{tabular}   \\
Post Condition        & The task is updated                                                                                                                                               
\end{tabular}
\linebreak
\\[1cm]
\begin{tabular}{l|l}

Name                  & UC03: Edit a task                                                                                                                                                                     \\
Description           & To edit an existing task                                                                                                                                                                                      \\
Pre Condition         & BoxIn is already running                                                                                                                                                                                      \\
Basic Course of Event & \begin{tabular}[c]{@{}l@{}}1. User types the command to edit an already existing task and \\specifying the relevant fields to be changed\\ 2. The program edits the task as per the user's command\end{tabular} \\
Alternative Path      & \begin{tabular}[c]{@{}l@{}}1. If the task does not exist, the program displays the relevant message.\\ 2. Prompts the user to re-enter the command.\end{tabular}                                              \\
Post Condition        & The task is updated                                                                                                                                                                                          
\end{tabular}
\linebreak
\\[1cm]
\begin{tabular}{l|l}
Name                  & UC04: Undo action                                                                                                                                   \\
Description           & To undo the previous command                                                                                                                                                \\
Pre Condition         & BoxIn is already running.                                                                                                                                                   \\
Basic Course of Event & 1. User types the command to undo the previous command.                                                                                                                     \\
Alternative Path      & \begin{tabular}[c]{@{}l@{}}1. If the previous action does not exist, the program displays the relevant \\message.\\ 2. Prompts the user to re-enter the command.\end{tabular} \\
Post Condition        & The most recent task is undone.                                                                                                                                                       
\end{tabular}
\linebreak
\\[1cm]
\begin{tabular}{l|l}

Name                  & UC05: Search task                                                                                                                                          \\
Description           & To search a task                                                                                                                                                                   \\
Pre Condition         & BoxIn is already running.                                                                                                                                                          \\
Basic Course of Event & \begin{tabular}[c]{@{}l@{}}1. User types the command to search for a task.\\ 2. The result is displayed.\end{tabular}                                                              \\
Alternative Path      & \begin{tabular}[c]{@{}l@{}}1. If the syntax does not match, prompts the user to re-enter the command.\\ 2. If the task does not exist, relevant message is displayed.\end{tabular} \\
Post Condition        & The tasks are displayed.                                                                                                                                                  
\end{tabular}
\linebreak
\\[1cm]
\begin{tabular}{l|l}
Name                  & UC06: Sort task                                                                                                  \\
Description           & To sort tasks                                                                                                                            \\
Pre Condition         & BoxIn is already running.                                                                                                                \\
Basic Course of Event & \begin{tabular}[c]{@{}l@{}}1. User types the command to sort the tasks.\\ 2. The program displays the task in sorted order.\end{tabular} \\
Alternative Path        & 1. Sort criteria is not specified and tasks are sorted using the default order.             	                                                                                                                       \\
Post Condition        & -                                                                                                                                       
\end{tabular}
\linebreak
\\[1cm]
\begin{tabular}{l|l}

Name                  & UC07: Display task                                                                                      \\
Description           & To display a task                                                                                                               \\
Pre Condition         & BoxIn is already running.                                                                                                       \\
Basic Course of Event & \begin{tabular}[c]{@{}l@{}}1. User types the command to display the task.\\ 2. The program displays the task.\end{tabular}      \\
Alternative Path      & \begin{tabular}[c]{@{}l@{}}1. The relevant task does not exist.\\ 2. Program prompts the user to re-enter command.\end{tabular} \\
Post Condition        & -                                                                                                                              
\end{tabular}


\section{Appendix B: Setting up}
To set up, you will need Windows Operating System, VS2012, Boost, and Git.
\subsection{Qt 5.3.1}
The Visual Studio plugin for Qt. You can find it at http://qt-project.org/downloads. 
Scroll to the bottom of the page and look for qt-vs-addin-1.2.3-opensource.exe. 
Then open Visual Studio. The top bar should show QT5 -> QT Options. Make 
sure that the correct version of QT is selected. 
Install Qt to C:/Qt
\subsection{Boost}
Boost libraries - version 1.57, vc2012 (vc11.0), 32 bit. You can find it at\\
\url{http://tinyurl.com/BoxInDevBoost}\\ and install to \texttt{C:/Boost}
\subsection{Visual Studio plugin}
The Visual Studio plugin for Qt. You can find it at http://qt-project.org/downloads. 
Scroll to the bottom of the page and look for qt-vs-addin-1.2.3-opensource.exe. 
Then open Visual Studio. The top bar should show QT5 -> QT Options. Make 
sure that the correct version of QT is selected. 
\subsection{Git}
You can download Git from Github.com and register as a member, then clone the repository, which gives you a local copy of the code.

\section{Appendix C: Testing Instructions}
Both unit tests and system tests are held in the same project file \texttt{(UnitTest)} and all code is contained in \texttt{unittest.cpp}, with headers in \texttt{unittest.h}. The 2 sets of tests are separated by comments indicating the start of unit tests.\\
The following code sample shows a simple test's syntax:\\
\texttt{TEST\_METHOD(ExtractCommand)\{\\
\indent SimpleParser parser;\\
\indent std::string expected = "add";\\
\indent Assert::AreEqual(parser.getField("add",TypeCommand), expected);\\
        \}}
\subsection{Setting up}
Change the directory of the include directory to match that of the project directory
\subsection{Prerequisites to testing}
To create any unit test that would include any GUI components, it is necessary to first start a QApplication. Refer to the following code example. \texttt{foo} is a defined function used to create the integer reference, but has no further use. The application need not be used further.\\
\texttt{int\& argc = foo();\\
char** argv;\\
QApplication app(argc, argv);
}
\subsection{Testing policy}
All newly added code should come together with a series of unit tests to prove it works for both the general and the borderline cases. Also, before committing any new code, it must clear all system level tests.

\subsection{Note on Qt}
It is possible to get the system to run events by using the emit function to generate te signals that normally would be generated from user activity


\section{Appendix D: Full Class Diagram}
\includegraphics[width=\textwidth, height=\textheight - 2cm]{class_diagram}

\end{document}